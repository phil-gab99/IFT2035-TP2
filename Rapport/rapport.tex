\documentclass[10pt, titlepage]{article}

\usepackage[margin=0.5in]{geometry}
\usepackage{amsmath, amssymb}
\usepackage{microtype}
\usepackage{titlesec}

% Code snippets
\usepackage{listings}
\usepackage{color}

\definecolor{dkgreen}{rgb}{0,0.6,0}
\definecolor{gray}{rgb}{0.5,0.5,0.5}
\definecolor{mauve}{rgb}{0.58,0,0.82}

\lstset{frame=tb,
    language=Haskell,
    showstringspaces=false,
    columns=flexible,
    basicstyle={\small\ttfamily},
    numbers=none,
    numberstyle=\tiny\color{gray},
    keywordstyle=\color{blue},
    commentstyle=\color{dkgreen},
    stringstyle=\color{mauve},
    breaklines=true,
    breakatwhitespace=true,
    tabsize=4
}

\titlespacing{\section}{0pt}{*1.5}{*1.5}

\begin{document}

\title{{\Huge \textbf{TP2 - Rapport}}}
\author{Philippe Gabriel \\ Yan Zhuang}
\date{21 juin 2021}

\maketitle

\pagenumbering{arabic}
\setcounter{page}{2}

\newpage

\section{Premier aperçu}
Comme première étape de travail, nous avions tenter d'apparier les données de
ce travail avec celui du premier travail pratique. Nous avions donc commencer
par supposer que les relations \texttt{infer} et \texttt{check} devrait
ressembler à aux fonctions du même nom dans le premier travail, et que les
relations \texttt{expand} et \texttt{coerce} ressemblait en quelque sorte à
\texttt{s2l}. Nous avions vite remarqué que de telles suppositions étaient
fausses à prendre, surtout dans la méthode d'implémentation car on remarque
dans le travail sur le langage \texttt{psil} que l'on peut diviser le travail
en quatre phases distinctes qui est même reflété par l'implémentation.
Cependant, pour le langage $\mu$\texttt{pts}, les relations communiquent toutes
entre elles en quelque sorte. Cela a rendu la tâche assez difficile au début.
La technique que l'on a employé par après pour ce travail est une sorte de
"Test-driven development" (tdd). On roulait les tests à l'aide de
\texttt{test\_samples} en appliquant l'outil de déboguage \texttt{trace}
qu'offre l'environnement de \texttt{prolog}. Cela nous permet de situer
l'emplacement des échecs d'exécution et nous permet par la suite de
corriger ceux-ci.

\section{Initialiser l'environnement}


\end{document}